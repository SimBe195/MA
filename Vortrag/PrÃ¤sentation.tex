\documentclass{beamer}
\usetheme{Warsaw}  %% Themenwahl

\usepackage[utf8]{inputenc}
\usepackage[Algorithmus]{algorithm}
\usepackage[noend]{algpseudocode}
\usepackage{multirow}
\usepackage{diagbox}
\usepackage{ stmaryrd }


% ========== Abkuerzungen ==========
\newcommand{\N}{\mathbb{N}}
\newcommand{\Z}{\mathbb{Z}}
\newcommand{\Q}{\mathbb{Q}}
\newcommand{\R}{\mathbb{R}}
\newcommand{\C}{\mathbb{C}}
\newcommand{\G}{\mathbb{G}}
\newcommand{\F}{\mathbb{F}}
\renewcommand{\P}{\mathbb{P}}
\newcommand{\B}{\mathcal{B}}
\newcommand{\No}{\mathcal{N}}
\newcommand{\Det}{\text{Det}}
\newcommand{\Min}{\text{Min}}
\renewcommand{\O}{\mathcal{O}}
\newcommand{\I}{\mathcal{I}}
\newcommand{\J}{\mathcal{J}}
\newcommand{\p}{\mathfrak{p}}
\newcommand{\q}{\mathfrak{q}}
\renewcommand{\a}{\mathfrak{a}}
\renewcommand{\b}{\mathfrak{b}}
\newcommand{\A}{\mathfrak{A}}
\newcommand{\Bfrak}{\mathfrak{B}}
\makeatletter
\renewcommand{\i}{\text{\expandafter\@slowromancap\romannumeral 1@}}
\newcommand{\ii}{\text{\expandafter\@slowromancap\romannumeral 2@}}
\makeatother
\newcommand{\Kern}{\text{Kern}}
\newcommand{\Bild}{\text{Bild}}
\newcommand{\ggT}{\text{ggT}}
\newcommand{\kgV}{\text{kgV}}
\newcommand{\Aut}{\text{Aut}}
\newcommand{\Dim}{\text{Dim}}

\title{Extremale Gitter mit großen Automorphismen}
\author{Simon Berger}
\date{21. September 2018}

\begin{document}

\begin{frame}[plain]
	\titlepage
\end{frame}

\begin{frame}[plain]
	\frametitle{Große Automorphismen}
	\begin{Definition}
		Sei $L$ ein $\Z$-Gitter der Dimension $n$. Ein \textit{großer Automorphismus} von $L$ ist ein $\sigma \in \Aut(L)$ von Ordnung $m \in \N$, sodass $\Phi_m \vert \mu_\sigma$ und $\frac{n}{2} < \varphi(m) \leq n$.
	\end{Definition}
	\pause
	\par
	$\rightarrow$ Ziel: Gitter mit großen Automorphismen klassifizieren.
\end{frame}

\begin{frame}[plain]
	\frametitle{Große Automorphismen}
	Sei $L$ Gitter mit großem Automorphismus $\sigma$ der Ordnung $m$ in Vektorraum $V$.
	\pause
	\[\Rightarrow V = \Kern(\frac{\mu_\sigma}{\Phi_m}(\sigma)) \perp \Kern(\Phi_m(\sigma))\]
	\[ M := L_1 \perp L_p := L \cap \Kern(\frac{\mu_\sigma}{\Phi_m}(\sigma)) \perp L \cap \Kern(\Phi_m(\sigma))\]
\end{frame}

\begin{frame}[plain]
	\frametitle{Große Automorphismen}
	\[ M := L_1 \perp L_p := L \cap \Kern(\frac{\mu_\sigma}{\Phi_m}(\sigma)) \perp L \cap \Kern(\Phi_m(\sigma))\]
	\begin{itemize}
		\item Falls ein $p \in \P$ existiert, sodass $\frac{\mu_\sigma}{\Phi_m} \mid (X^\frac{m}{p} - 1)$, dann ist $L_1 = L \cap \Kern(\sigma^\frac{m}{p}-1)$ das Fixgitter und $L_p = L \cap \Bild(\sigma^\frac{m}{p} - 1)$ das Bildgitter eines Automorphismus von Primzahlordnung.
	\end{itemize}
\end{frame}

\begin{frame}[plain]
	\frametitle{Strategie zur Klassifikation}
	\begin{itemize}
		\item Primteiler $p$ von $m$ mit $\ggT(p,\ell) = 1$ durchgehen.
		\pause
		\item Mögliche Automorphismentypen der Ordnung $p$ durchgehen.
		\pause
		\item Mögliche Bildgitter aufzählen.
		\pause
		\item Mögliche Fixgitter aufzählen.
		\pause
		\item Kandidaten für $\sigma$ aufzählen.
		\pause
		\item $L$ als $\sigma$-invariantes Obergitter von $L_1 \perp L_p$ konstruieren.
	\end{itemize}
\end{frame}

\section{Automorphismen von Primzahlordnung}
\frame{\tableofcontents[currentsection]}
\begin{frame}[plain]
	\frametitle{Automorphismen von Primzahlordnung}
	\begin{Satz}
		Sei $L$ gerade, $n$-dim. von q.-freier Stufe $\ell$, $\Det(L) = \ell^k$. Sei zudem $\sigma \in \Aut(L)$ von Typ $p - (n_1, n_p) - s - q_1 - (k_{1,1}, k_{p,1}) - \dots$, wobei $\ggT(p, \ell) = 1$. Dann gilt:
		\begin{itemize}
			\item $n_1 + n_p = n$.
			\item $s \in \lbrace 0, \dots, \min(n_1, \frac{n_p}{p-1}) \rbrace$.
			\item $s \equiv_2 \frac{n_p}{p-1}$ und für $p = 2$ zusätzlich $s \equiv_2 0$.
			\item $k_{1,i} \in \lbrace 0, \dots, \min(n_1, k) \rbrace$.
			\item $k_{1,i} \equiv_2 k$.
			\item $k_{p,i} \in \lbrace 0, \dots, \min(n_p, k) \rbrace$.
			\item $k_{p,i} \equiv_2 0$.
			\item $\left( 2f(q_i)\right) \vert k_{p,i}$, wobei $f(q_i)$ den Trägheitsgrad von $q_i \Z_{\Q(\zeta_p + \zeta_p^{-1})}$ bezeichne.
			\item $k_{1,i} + k_{p,i} = k$.
		\end{itemize}
	\end{Satz}
\end{frame}

\section{Bildgitter}
\frame{\tableofcontents[currentsection]}
\begin{frame}[plain]
	\frametitle{Bildgitter}
	Da $\varphi(m) > \frac{n}{2}$ ist $\Kern(\Phi_m)$ ein eindimensionaler $\Q(\zeta_m)$-Vektorraum $\rightsquigarrow$ $L_p$ ist Ideal-Gitter.\par
	\pause
	$\Rightarrow$ Effizient berechenbar mit dem Algorithmus aus dem ersten Teil!
\end{frame}

\section{Fixgitter}
\frame{\tableofcontents[currentsection]}
\begin{frame}[plain]
	\frametitle{Fixgitter}
	\begin{itemize}
		\item Für das Fixgitter kennen wir Dimension, Determinante und Stufe.
		\pause
	 	\item $\Rightarrow$ Finden wir nur mittels Geschlechteraufzählung.
		\pause 
		\item Falls $p > 2$ und $\ell$ prim, kennen wir genau das Geschlechtssymbol von $L_1$.
		\pause
		\item Ansonsten zumindest die Elementarteiler.
		\pause 
		\item Nach Konstruktion von Vertretern mit passenden Elementarteilern (David Lorch) Aufzählung des gesamten Geschlechts mit der Kneser'schen Nachbarmethode.
	\end{itemize}
\end{frame}

\begin{frame}[plain]
	\frametitle{Fixgitter}
	\begin{Satz}
		Sei $L$ ein Gitter von Dimension $\geq 3$. Hat für jede Primzahl $q \in \P$ die Jordanzerlegung von $\Z_q \otimes L$ mindestens eine Komponente von Dimension $\geq 2$, so besteht der Nachbarschafts-Graph von $L$ aus genau einer Zusammenhangskomponente.
	\end{Satz}
	\pause
	\begin{itemize}
		\item Schwache Bedingung ist beinahe immer erfüllt.
		\item Wir erhalten durch sukzessive Nachbarbildung das gesamte Geschlecht.
		\item Benutzen als Abbruchbedingung das Maß des Geschlechtes.
	\end{itemize}
\end{frame}

\section{Kandidaten für $\sigma$}
\frame{\tableofcontents[currentsection]}
\begin{frame}[plain]
	\frametitle{Kandidaten für $\sigma$}
	\begin{itemize}
		\item $\sigma$ operiert auf den Faktorgruppen $L_1^{\#,p} / L_1$ und $L_p^{\#,p} / L_p$.
		\pause
		\item Die Faktorgruppen sind isomorph als $\F_p[\sigma]$-Moduln.
		\pause
		\item $\Rightarrow$ Minimalpolynome der Operationen von $\sigma$ auf den beiden Faktorgruppen sind identisch.
		\pause
		\item Das Minimalpolynom auf den Faktorgruppen ist $\Phi_\frac{m}{p}$, falls $[L : M] > 1$.
		\pause		
		\item Wähle Vertreter $\sigma_1$ und $\sigma_p$ der Konjugiertenklassen der Automorphismen von $L_1$ und $L_p$, die mit dem richtigen Minimalpolynom auf $L_1^{\#,p} / L_1$ und $L_p^{\#,p} / L_p$ operieren. Setze $\sigma := \text{diag}(\sigma_1, \sigma_p)$.
	\end{itemize}
\end{frame}

\section{Konstruktion von Obergittern}
\frame{\tableofcontents[currentsection]}
\begin{frame}[plain]
	\frametitle{Konstruktion von Obergittern}
	\begin{itemize}
		\item Die ganzen Obergitter von $M$ mit Index $p^s$ haben die Form
		\[L_\varphi := \lbrace (x_1,x_p) \in L_1^{\#,p} \perp L_p^{\#,p} \mid \varphi(x_1+L_1) = x_p + L_p \rbrace\]
		für die Isometrien $\varphi : (L_1^{\#,p} / L_1, \overline{b_1}) \rightarrow (L_p^{\#,p} / L_p, -\overline{b_p})$
		\pause
		\item Damit $L_\varphi$ invariant unter $\sigma = \text{diag}(\sigma_1, \sigma_p)$ ist, muss $\varphi \circ \sigma_1 = \sigma_p \circ \varphi$ gelten.
		\pause
		\item Für $c \in C_{\text{Aut}(L_1)}(\sigma_1)$ ist $L_{\varphi} \cong L_{\varphi c}$.
		\pause
		\item Damit können wir die Obergitter aufzählen, indem wir die relevanten Isometrien modulo $C_{\text{Aut}(L_1)}(\sigma_1)$ durchgehen.
		\pause
		\item Bemerkung: In vielen Fällen können wir einfach alle ganzen Obergitter von $M$ mit Index $p^s$ aufzählen, ohne auf $\sigma$-Invarianz zu achten; so erhalten wir ggf. noch mehr Gitter!
	\end{itemize}
\end{frame}

\begin{frame}[plain]
	\frametitle{Finaler Algorithmus}
	Alle Teilschritte können nun zu einem Algorithmus zusammengesetzt werden.
\end{frame}

\begin{frame}[plain]
\begin{tiny}
\begin{table}[H]
	\centering
	\begin{tabular}{|c||c|c|c|c|c|c|c|c|c|c|c|}
		\hline
		\backslashbox{$n$}{$\ell$}	&$1$	&$2$	&$3$	&$5$	&$6$	&$7$	&$11$	&$14$	&$15$	&$23$\\ \hline \hline
		$2$		&$-$		&$-$		&$1$		&$-$		&$-$		&$-$		&$-$		&$-$		&$-$	&$-$\\ \hline
		$4$		&$-$		&$1$		&$1$		&$-$		&$-$		&$-$		&$1$		&$1$		&$-$	&$1$\\ \hline
		$6$		&$-$		&$-$		&$1$		&$-$		&$-$		&$1$		&$1$		&$-$		&$-$	&$-$\\ \hline
		$8$		&$1$		&$1$		&$1$		&$1$		&$1$		&$1$		&$1$		&$1$		&$1$	&$-$\\ \hline
		$10$	&$-$		&$-$		&$1$		&$-$		&$-$		&$-$		&$1$		&$-$		&$-$	&$-$\\ \hline
		$12$	&$-$		&$2$		&$1$		&$1$		&$1$		&$-$		&$-$		&$1$		&$1$	&$-$\\ \hline
		$14$	&$-$		&$-$		&$1$		&$-$		&$-$		&$-$		&$-$		&$-$		&$-$	&$-$\\ \hline
		$16$	&$2$		&$1$		&$2$		&$-$		&$1$		&$3$		&$-$		&$-$		&$1$	&$-$\\ \hline
		$18$	&$-$		&$-$		&$1$		&$-$		&$-$		&$-$		&$-$		&$-$		&$-$	&$-$\\ \hline
		$20$	&$-$		&$1$		&$3$		&$-$		&$1$		&$-$		&$-$		&$-$		&$-$	&$-$\\ \hline
		$22$	&$-$		&$-$		&$2(1^\ast)$&$-$		&$-$		&$-$		&$-$		&$-$		&$-$	&$-$\\ \hline
		$24$	&$1$		&$8(2^\ast)$&$1$		&$1$		&$5(3^\ast)$&$-$		&$-$		&$-$		&$-$	&$-$\\ \hline
		$26$	&$-$		&$-$		&$2$		&$-$		&$-$		&$-$		&$-$		&$-$		&$-$	&$-$\\ \hline
		$28$	&$-$		&$35(25^\ast)$&$3(2^\ast)$&$-$		&$-$		&$-$		&$-$		&$-$		&$-$	&$-$\\ \hline
		$30$	&$-$		&$-$		&$-$		&$-$		&$-$		&$-$		&$-$		&$-$		&$-$	&$-$\\ \hline
		$32$	&$-$		&$2$		&$67(65^\ast)$&$-$		&$-$		&$-$		&$-$		&$-$		&$-$	&$-$\\ \hline
		$34$	&$-$		&$-$		&$-$		&$-$		&$-$		&$-$		&$-$		&$-$		&$-$	&$-$\\ \hline
		$36$	&$-$		&$-$		&$-$		&$-$		&$-$		&$-$		&$-$		&$-$		&$-$	&$-$\\ \hline
	\end{tabular}
	\caption{Anzahl der durch den Algorithmus konstruierten extremalen stark $\ell$-modularen Gitter in Dimension $n \leq 36$ sowie ggf. der Anzahl der bisher unbekannten Gitter darunter}
\end{table}
\end{tiny}
\end{frame}

\begin{frame}[plain]
	\frametitle{Vollständigkeit}
	\begin{itemize}
	\item Erinnerung: Es muss ein $p \in \P$ mit $\ggT(p, \ell) = 1$ existieren, sodass $\frac{\mu_\sigma}{\Phi_m} \mid (X^\frac{m}{p} - 1)$\\
	\pause
	\item $\rightsquigarrow$ Wie stark ist diese Voraussetzung?\\
	\pause
	\item Dazu: Gitter charakterisieren, die \textbf{nicht} auf diese Weise konstruiert werden können.
	\end{itemize}
\end{frame}

\begin{frame}[plain]
	\frametitle{Vollständigkeit - Beispiel}
	Sei $\ell = 3$, $n = 24$.\\
	\pause
	Die möglichen Automorphismentypen von Ordnung $\in \P_{\neq 3}$ sind:\\
	\begin{align*}
		&2 - (12, 12) - 12 - (6, 6)		&(1 \text{ Fixgitter})\\
		&2 - (0, 24) - 0 - (0, 12)		&(1 \text{ Fixgitter})\\
		&5 - (8, 16) - 4 - (8, 4)		&(5 \text{ Fixgitter})\\
		&5 - (8, 16) - 4 - (4, 8) 		&(4 \text{ Fixgitter})\\
		&5 - (0, 24) - 0 - (0, 12) 		&(1 \text{ Fixgitter})\\
		&7 - (0, 24) - 0 - (0, 12) 		&(1 \text{ Fixgitter})\\
		&11 - (4, 20) - 2 - (2, 10) 	&(1 \text{ Fixgitter})\\
		&13 - (0, 24) - 0 - (0, 12) 	&(1 \text{ Fixgitter})
	\end{align*}
	\pause
	Für $12 < \varphi(m) \leq 24$ und da $m$ keine Primteiler $> 13$ hat: $m \in \lbrace 25, 27, 32, 33, 40, 44, 45, 48, 50, 54, 60, 66, 72, 84, 90\rbrace$.
\end{frame}

\begin{frame}[plain]
	\frametitle{Vollständigkeit - Beispiel}
	\begin{itemize}
		\item[$m = 25$:] $\Phi_{25} \mid \mu_\sigma$, $\frac{\mu_\sigma}{\Phi_{25}} \vert (X^5-1)$ $\Rightarrow$ wird von Alg. gefunden.
		\pause
		\item[$m = 48$:] Ang. $\sigma^{24}$ hat Typ $2 - (12, 12) - 12 - (6, 6)$, dann $\Phi_{48} \nmid \mu_\sigma$. Somit $\Phi_{16} \mid \mu_\sigma$ und $\mu_\sigma \mid (X^{24}-1) \Phi_{16}$.\\
		$\Rightarrow$ $\Bild(\sigma^{24}-1)$ ist $\Q(\zeta_{16})$-VR, aber $\Dim(\Bild(\sigma^{24}-1)) = 12 \lightning$\\
		$\Rightarrow \mu_\sigma \in \lbrace \Phi_{16} \Phi_{48}, \Phi_{48} \rbrace$. Für $\mu_\sigma = \Phi_{48}$ ist $L$ aber ein Ideal-Gitter über $\Q(\zeta_{48})$ und wird gefunden.\\
		\pause
		\begin{center} \vdots \end{center}
	\end{itemize}
\end{frame}

\begin{frame}[plain]
	\frametitle{Vollständigkeit - Methoden}
	Sei $L$ ein Gitter mit einem Automorphismus $\sigma$ der Ordnung $m$, sodass $\frac{n}{2} < \varphi(m) \leq n$, aber $L$ kann nicht durch den Algorithmus gefunden werden.\\
	\pause
	Betrachte die charakteristischen Polynome
	\[\chi_\sigma := \Phi_{d_1}^{c_1} \dots \Phi_{d_k}^{c_k} \]
	für die Teiler $d_1 < d_2 < \dots < d_k$ von $m$.\\
	\pause
	\begin{itemize}
		\item Für Ordnung $m$: $\kgV\lbrace d_i \vert c_i > 0 \rbrace \stackrel{!}{=} m$.\\
		\pause
		\item Wenn
	 		\[c_k = 1 \text{ und } \kgV \lbrace d_i \mid i \in \lbrace 1, \dots, k-1 \rbrace \text{ und } c_i > 0 \rbrace \mid \frac{m}{p}\]
			für ein $p \in \P$, $\ggT(p, \ell) = 1$ erfüllt ist, wird $L$ gefunden.
	\end{itemize}
\end{frame}

\begin{frame}[plain]
	\frametitle{Vollständigkeit - Methoden}
	Kennt man für eine Menge von Primteilern $p_1, \dots, p_t \mid m$ die Typen
	\begin{align*}
		p_1 - &(n_{1,1}, \dots\\
		&\vdots\\
		p_t - &(n_{t,1}, \dots
	\end{align*}
	der Automorphismen $\sigma^\frac{m}{p_1}, \sigma^\frac{m}{p_2}, \dots, \sigma^\frac{m}{p_t}$, so muss $c := (c_1, \dots, c_k)$ eine Lösung von $c M= (n_{1,1}, n_{2,1}, \dots, n_{t,1}, n)$ mit der Matrix
	\[M \in \N_0^{k \times (t+1)}, \quad M_{i,j} := \begin{cases} \varphi(d_i)	&, d_i \mid \frac{m}{p_j} \text{ oder } j = t+1\\ 0 &, \text{sonst}\end{cases}\]
sein.
\end{frame}

\begin{frame}[plain]
	\frametitle{Vollständigkeit - Methoden}
	Für $p \mid m$ ist 
	\[\vert \sigma_1 \vert = \kgV \lbrace d_i \mid \nu_p(d_i) < \nu_p(m) \text{ und } c_i > 0 \rbrace.\]
	\[\vert \sigma_p \vert = \kgV \lbrace d_i \mid \nu_p(d_i) = \nu_p(m) \text{ und } c_i > 0 \rbrace.\]
	\pause
	$\rightsquigarrow$ Wenn für $\sigma^\frac{m}{p}$ alle möglichen Fix- oder Bildgitter aufgezählt werden können, muss mindestens eines davon einen Automorphismus der passenden Ordnung haben.
\end{frame}

\begin{frame}[plain]
	\frametitle{Vollständigkeit - Methoden}
	Seien $p_1, p_2 \mid m$.\\
	Wenn für alle Faktoren $\Phi_{d_i} \mid \chi_\sigma$, immer \textit{entweder} $\nu_{p_1}(d_i) = \nu_{p_1}(m)$ \textit{oder} $\nu_{p_2}(d_i) = nu_{p_2}(m)$ gilt, induzieren $p_1$ und $p_2$ dasselbe Teilgitter.\\
	\pause
	$\rightsquigarrow$ Index $1$!\\
	\pause
	z.B: $\mu_\sigma = \Phi_{12} \Phi_{20}$, dann induzieren $\sigma^{12}$ und $\sigma^{20}$ dasselbe Teilgitter.
\end{frame}

\begin{frame}[plain]
	\frametitle{Vollständigkeit - Beispiel}
	\begin{Satz}
		Sei $L$ ein extremales $3$-modulares Gitter in einem bilinearen Vektorraum $(V,b)$ der Dimension $24$. Dann hat $L$ keine Automorphismen der Ordnung $7$ sowie der Ordnung $p \in P_{\ge 13}$. Ist $\sigma \in \Aut(L)$ von Ordnung $m$ mit $12 < \varphi(m) \leq 24$, so ist\linebreak
		$m \in \lbrace 27, 33, 48, 54, 60, 66, 72\rbrace$. Außerdem gelten folgende Einschränkungen:
		\begin{itemize}
			\item $m = 27 \Rightarrow \Phi_{27} \vert \mu_\sigma$
			\item $m = 33 \Rightarrow \chi_\sigma = \Phi_1^2 \Phi_3 \Phi_{11}^2$
			\item $m = 48 \Rightarrow \chi_\sigma = \Phi_{16} \Phi_{48}$
			\item $m = 54 \Rightarrow \Phi_{24} \vert \mu_\sigma$ und $\sigma^{27}$ hat Typ $2 - (0, 24) - 0 - (0, 12)$.
			\item $m = 60 \Rightarrow \chi_\sigma = \Phi_4^2 \Phi_{12} \Phi_{20}^2$
			\item $m = 66 \Rightarrow \mu_\sigma \vert \Phi_{2} \Phi_{6} \Phi_{22} \Phi_{66}$
			\item $m = 72 \Rightarrow \Phi_8 \vert \mu_\sigma$
		\end{itemize}
	\end{Satz}
\end{frame}


\begin{frame}[plain]
	\frametitle{Vollständigkeit - Methoden}
	Methoden zur Analyse der charakteristischen Polynome in \texttt{MAGMA} implementiert und verschiedene $\ell$ und $n$ ausgewertet.\\
\end{frame}

\begin{frame}[plain]
	\frametitle{Ende}
	\begin{large} \begin{center} Vielen Dank für eure Aufmerksamkeit! \end{center} \end{large}
\end{frame}

\end{document}