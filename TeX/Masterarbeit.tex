\documentclass[12pt,a4paper,halfparskip,headsepline, bibtotocnumbered]{scrreprt}

% ========== Pakete einbinden ==========
\usepackage[utf8]{inputenc}          % um Umlaute direkt eingeben zu können
\usepackage[T1]{fontenc}             % für "modernere" Schriftkodierung
\usepackage[ngerman]{babel}          % Sprachanpassungen für neue Deutsche Rechtschreibung
\usepackage[intlimits]{amsmath}      % für erweiterte mathematische Konstrukte
\usepackage{amsfonts}                % für erweiterte mathematische Schriften
\usepackage{amssymb}                 % für erweiterten Symbolvorrat
\usepackage[sc]{mathpazo}            % Schriftart Palatino
\usepackage{textcomp}                % für erweiterten Text-Symbolvorrat
\usepackage{typearea}                % für die Berechnung der Seitenründer
\usepackage{scrpage2}                % für Kopf- und Fußzeilen
\usepackage{array}                   % für erweiterte Tabellensatzmöglichkeiten
\usepackage{enumerate}               % für komfortablere Aufzählungen
\usepackage{titlesec}                % für Kontrolle der Abschnittüberschriften
\usepackage[amsmath, thmmarks]{ntheorem} % für Definitionen, Sätze, ...
\usepackage{graphicx}
\usepackage{framed}
\usepackage{lmodern}
\usepackage{blindtext}
\usepackage{pdfpages}
\usepackage{url}
\usepackage{enumitem}

% Spezialpakete
\usepackage{fp}
\usepackage{tikz}
\usepackage{xcolor}
% TikZ-Bibliotheken
\usetikzlibrary{arrows}
\usetikzlibrary{shapes}
\usetikzlibrary{decorations.pathmorphing}
\usetikzlibrary{decorations.pathreplacing}
\usetikzlibrary{decorations.shapes}
\usetikzlibrary{decorations.text}
\usetikzlibrary{calc}

\usepackage{shadethm}

% ========== Layout ==========
\typearea{12}
\linespread{1.5}
\titleformat{\section}{\normalfont\normalcolor\Large\bfseries\centering}%
 {\S\,\thesection}{.66em}{}
\newcommand{\subsectiontitel}[1]{\raisebox{0.6ex}{\rule{12pt}{0.4pt}}%
 \hspace*{0.2cm}\textit{#1}\hspace*{0.2cm}\raisebox{0.6ex}{\rule{12pt}{0.4pt}}}
\titleformat{\subsection}[block]{\normalfont\normalcolor\centering}{}{0em}{\subsectiontitel}

% ---------- Definitionen, Sätze, ... ----------

\makeatletter

\newtheoremstyle{nummermitklammern}%
 {\item[\rlap{\vbox{\hbox{\hskip\labelsep \theorem@headerfont
  (##2)\ ##1\theorem@separator}\hbox{\strut}}}]}%
 {\item[\rlap{\vbox{\hbox{\hskip\labelsep \theorem@headerfont
  (##2)\ ##1 (##3)\theorem@separator}\hbox{\strut}}}]}%
\makeatother

\theoremstyle{nummermitklammern}
\theorembodyfont{\rmfamily}
%\theoremsymbol{\ensuremath{\diamond}}
\newtheorem{defsatzusw}{}[section]
\newtheorem{definition}[defsatzusw]{Definition}
\newtheorem{bezeichnung}[defsatzusw]{Bezeichnung}
\newtheorem{bezeichnungen}[defsatzusw]{Bezeichnungen}
\newtheorem{voraussetzung}[defsatzusw]{Voraussetzung}
\newtheorem{voraussetzungen}[defsatzusw]{Voraussetzungen}
\newtheorem{satz}[defsatzusw]{Satz}


\newtheorem{lemma}[defsatzusw]{Lemma}
\newtheorem{algorithmus}[defsatzusw]{Algorithmus}
\newtheorem{korollar}[defsatzusw]{Korollar}
\newtheorem{folgerung}[defsatzusw]{Folgerung}
\newtheorem{hilfssatz}[defsatzusw]{Hilfssatz}
\newtheorem{proposition}[defsatzusw]{Proposition}
\newtheorem{bemerkung}[defsatzusw]{Bemerkung}
\newtheorem{bemerkungen}[defsatzusw]{Bemerkungen}
\newtheorem{beispiel}[defsatzusw]{Beispiel}
\newtheorem{beispiele}[defsatzusw]{Beispiele}
\theoremstyle{nonumberbreak}
\newtheorem{beweis2}{Beweis (Anfang):}
\theoremsymbol{\ensuremath{\square}}
\newtheorem{beweis}{Beweis:}
\newtheorem {fazit} {Fazit}



\definecolor{shadethmcolor}{rgb}{.9,.9,.9}   % Farbe des Hintergrundes 
\definecolor{shaderulecolor}{rgb}{0.0,0.0,1.0}   % Farbe des Rahmens
\setlength{\shadeboxrule}{1pt}   % Breite des Rahmens
\newshadetheorem{sthm}{lemma}
\newenvironment{thm}[1][]{%
  \definecolor{shadethmcolor}{rgb}{.9,.9,.9}%
  \definecolor{shaderulecolor}{rgb}{1.0,0.0,0.0}%
  \setlength{\shadeboxrule}{1pt}%
  \begin{sthm}[#1]%
}{\end{sthm}}

% ---------- Mathematik-Umgebungen ----------
\newenvironment{mathex}[1][LLLLLLLLLLLL]{
  \[%
  \newcolumntype{L}{>{\displaystyle\setlength{\arraycolsep}{4pt}}l}%
  \newcolumntype{C}{>{\displaystyle\setlength{\arraycolsep}{4pt}}c}%
  \newcolumntype{R}{>{\displaystyle\setlength{\arraycolsep}{4pt}}r}%
  \setlength{\arraycolsep}{1.5pt}%
  \begin{array}{>{\vspace*{1.3ex}}#1}%
}{
  \end{array}%
  \vspace*{-1.3ex}%
  \]%
  \ignorespacesafterend%
}

% ========== Abkuerzungen ==========
\newcommand{\N}{\mathbb{N}}
\newcommand{\Z}{\mathbb{Z}}
\newcommand{\Q}{\mathbb{Q}}
\newcommand{\R}{\mathbb{R}}
\newcommand{\C}{\mathbb{C}}
\newcommand{\G}{\mathbb{G}}
\newcommand{\F}{\mathbb{F}}
\newcommand{\B}{\mathcal{B}}
\newcommand{\No}{\mathcal{N}}
\newcommand{\Det}{\text{Det}}
\newcommand{\Min}{\text{Min}}
\renewcommand{\O}{\mathcal{O}}
\newcommand{\I}{\mathcal{I}}
\newcommand{\p}{\mathfrak{p}}
\renewcommand{\a}{\mathfrak{a}}
\renewcommand{\b}{\mathfrak{b}}

% ========== Sonstiges ==========
\renewcommand{\emptyset}{\text{\usefont{OMS}{cmsy}{m}{n}\symbol{59}}}
\bibliographystyle{alpha}
\allowdisplaybreaks


\begin{document}

%===================================================================
% Titelseite
%===================================================================

\begin{titlepage}
	\centering
	%\includegraphics[width=0.5\textwidth]{RWTH_Logo.png}\par\vspace{1cm}
	\vspace*{2cm}
	{\Large\bfseries Extremale Gitter mit großen Automorphismen\par}
	\vspace{1cm}
	{\scshape\Large Masterarbeit\par}
	\vspace{2cm}
	{\Large\itshape von Simon Berger\par}
	\vfill
	Vorgelegt am Lehrstuhl D für Mathematik der RWTH-Aachen University\par 
	bei Prof. Dr. Gabriele Nebe (1. Prüferin)\\ 
	und Prof. Dr. Markus Kirschmer (2. Prüfer)\par
	\vfill

% Bottom of the page
	{\large \today\par}
\end{titlepage}

%===================================================================
% Inhaltsverzeichnis
%===================================================================

\tableofcontents

%===================================================================
% Kapitel 1: Einleitung
%===================================================================
\chapter{Einleitung}


%===================================================================
% Kapitel 2: Definitionen
%===================================================================

\chapter{Grundbegriffe}

\section{Quadratische Vektorräume}

Wir wiederholen zunächst einige wichtige Begriffe aus der Gittertheorie, welche wir in der Arbeit häufig benötigen werden. Zunächst führen wir das Konzept eines quadratischen Vektorraumes ein. Die nun angeführten Definitionen sind \cite[Def. (2.1)]{kneser} entnommen.

\begin{framed}
	\begin{definition}
		\begin{enumerate}[label=(\roman*)]
			\item Sei $A$ ein Ring und $E$ ein $A$-Modul. Für eine symmetrische Bilinearform\linebreak
				$b : E \times E \rightarrow A$ heißt das Paar $(E,b)$ ein \textit{bilinearer $A$-Modul} (bzw. falls $A$ Körper ein \textit{bilinearer $A$-Vektorraum}).
			\item Eine Abbildung $q : E \rightarrow A$ mit den Eigenschaften
				\begin{align*}
					q(ax) = a^2 q(x)\quad \text{für } a \in A, x \in E\\
					q(x+y) = q(x) + q(y) + b_q(x,y)
				\end{align*}
				mit einer symmetrischen Bilinearform $b_q$ heißt \textit{quadratische Form}. Ein solches Paar $(E,q)$ heißt \textit{quadratischer $A$-Modul} (bzw. falls $A$ Körper ein \textit{quadratischer $A$-Vektorraum}).
			\item Eine \textit{isometrische Abbildung} (oder kurz \textit{Isometrie}) zwischen zwei bilinearen Moduln $(E,b)$ und $(E', b')$ ist ein Modulisomorphismus $f : E \rightarrow E'$ mit $b(x,y) = b'(f(x), f(y))$.
			\item Analog ist eine Isometrie zwischen zwei quadratischen Moduln $(E,q)$ und $(E', q')$ ist ein Modulisomorphismus $f : E \rightarrow E'$ mit $q(x) = q'(f(x))$ für alle $x \in E$.
		\end{enumerate}
	\end{definition}
\end{framed}

\begin{bemerkung}
	Auf einem quadratischen $A$-Modul $(E,q)$ ist $b_q : E \times E \rightarrow A, (x,y) \mapsto q(x+y) - q(x) - q(y)$ eine symmetrische Bilinearform. Andersherum erhält man aus jeder symmetrischen Bilinearform $b$ auf $E$ eine quadratische Form $q_b : E \rightarrow A, x \mapsto b(x,x)$. Es ist dabei $b_{q_b} = 2b$ und $q_{b_q} = 2q$. Ist $2 \in A^*$, so kann man daher stattdessen $q_b : E \rightarrow A, x \mapsto \frac{1}{2} b(x,x)$ wählen, womit die Konzepte der quadratischen Formen und der symmetrischen Bilinearformen auf $E$ vertauschbar sind.
\end{bemerkung}

Nun folgen Definitionen zum Gitterbegriff, zu finden in \cite[Def. (14.1), (14.2)]{kneser}.
\begin{framed}
	\begin{definition}
		\begin{enumerate}[label=(\roman*)]
			\item Sei $K$ ein Körper, $V$ ein endlich-dimensionaler $K$-Vektorraum mit Basis $(b_1,\dots,b_n)$. Ein $R$-Gitter in $V$ ist ein $R$-Untermodul $L$ von $V$, zu dem Elemente $a,b \in K^*$ existieren mit $a \sum_{i=1}^n R b_i \subseteq L \subseteq b \sum_{i=1}^n R b_i$.
			\item Sei $b$ eine nicht-ausgeartete symmetrische Bilinearform auf $V$ und $L$ ein Gitter in $V$. Dann ist auch $L^\# := \lbrace x \in V \mid b(x,y) \in R \text{ für alle } y \in L \rbrace$ ein $R$-Gitter und heißt \textit{das zu $L$ duale Gitter} (bzgl. $b$).
			\item Für $m \in \N$ heißt das Gitter $L^{\#,m} := \frac{1}{m}L \cap L^\#$ \textit{partielles Dualgitter} von $L$.
			\item Sei $(V,b)$ ein bilinearer $K$-Vektorraum und $L$ ein Gitter in $V$. Die Gruppe $\text{Aut}(L) := \lbrace \sigma : V \rightarrow V \mid \sigma \text{ ist Isometrie und } \sigma(L) = L \rbrace$ heißt die \textit{Automorphismengruppe} von $L$.
		\end{enumerate}
	\end{definition}
\end{framed}

\begin{bemerkung}
	Falls $R$ ein Hauptidealbereich ist, vereinfacht sich die Definition erheblich, da Teilmoduln von endlich erzeugten freien Moduln über Hauptidealbereichen wieder frei sind. Ein $R$-Gitter ist per Definition zwischen zwei freien Moduln eingespannt, also sind die $R$-Gitter in diesem Fall genau die freien $R$-Moduln von Rang $n$.
\end{bemerkung}

Insbesondere interessieren uns $\Z$-Gitter in $\R^n$. Für eben solche folgen nun ein paar weitere Definitionen, abgeleitet aus \cite[Def. (1.7), (1.13), (14.7), (26.1)]{kneser}.

\begin{framed}
	\begin{definition}
		Sei $L$ ein $\Z$-Gitter mit Basis $B = (e_1, \dots, e_n)$ in $(\R^n, b)$, für eine symmetrische Bilinearform $b$.
		\begin{enumerate}[label=(\roman*)]
			\item Die Matrix $G := \text{Gram}(B) = \left(b(e_i, e_j)\right)_{i,j=1}^n$ heißt \textit{Gram-Matrix} von $L$, $\Det(L) := \Det(G)$ heißt die \textit{Determinante} von $L$.
			\item Das Gitter $L$ heißt \textit{ganz}, falls $b(L,L) \subseteq \Z$ gilt.
			\item Das Gitter $L$ heißt \textit{gerade}, falls $b(x,x) \in 2\Z$ für alle $x \in L$ gilt.
			\item Die \textit{Stufe} von $L$ ist die kleinste Zahl $\ell \in \N$, sodass $\sqrt{\ell} L^\#$ ein gerades Gitter ist.
			\item Das \textit{Minimum} von $L$ ist definiert als $\Min(L) := \text{min} \lbrace b(x,x) \mid 0 \neq x \in L \rbrace$.
		\end{enumerate}
	\end{definition}
\end{framed}


\begin{bemerkung}\label{bem:dual}
	\begin{enumerate}[label=(\roman*)]
	\item Nach \cite[Satz (14.7)]{kneser} gilt $\Det(L) = \vert L^\# / L \vert$. 
		Insbesondere ist die Determinante für $\Z$-Gitter unabhängig von der Wahl der Basis. Allgemeiner ist die Determinante von $R$-Gittern modulo $\left( R^\ast \right) ^2$ eindeutig bestimmt \cite[(1.13)]{kneser}.
	\item Direkt aus der Definition des dualen Gitters folgt: $L$ ist ganz genau dann, wenn $L \subseteq L^\#$.
	\item Ein gerades Gitter $L$ ist notwendigerweise ganz, denn seien $x,y \in L$, dann ist
		\begin{equation*}
			b(x,y) = \frac{b(x+y,x+y) - b(x,x) - b(y,y)}{2} \in \Z.
		\end{equation*}
	\item Ist $B = (e_1,\dots,e_n)$ eine Basis von $L$, dann ist $B^\ast := (e_1^\ast, \dots, e_n^\ast)$, wobei $b(e_i, e_j^\ast) = \delta_{ij}$, eine Basis von $L^\#$. Es gilt $\text{Gram}(B^\ast) = \text{Gram}(B)^{-1}$ \cite[(1.14)]{kneser}.
	\end{enumerate}
\end{bemerkung}

Da wir uns im Zuge dieser Arbeit in der Regel mit geraden Gittern quadratfreier Stufe beschäftigen, ist das folgende Lemma aus \cite[Lemma 1.1.1]{juergens} von großer Bedeutung.

\begin{framed}
	\begin{lemma}\label{lem:level}
		Sei $L$ ein gerades Gitter der Stufe $\ell$, wobei $\ell$ quadratfrei. Dann ist $\ell$ gleichzeitig die kleinste natürliche Zahl $a$, sodass $a L^\# \subseteq L$ (also der Exponent der Diskriminantengruppe $L^\# / L$).
	\end{lemma}
\end{framed}

\section{Modulare Gitter}

Wir kommen nun zum ursprünglich von Quebbemann eingeführten Konzept \textit{modularer Gitter} \cite{quebbemann}. Die hier verwendete Definition ist in \cite{fluckiger} zu finden.

\begin{framed}
	\begin{definition}
		Sei $L$ ein gerades Gitter und $\ell \in \N$.
		\begin{enumerate}[label=(\roman*)]
			\item $L$ heißt \textit{$\ell$-modular}, falls $L \cong \sqrt{\ell}L^\#$.
			\item $L$ heißt \textit{stark $\ell$-modular}, falls $L \cong \sqrt{m} L^{\#,m}$ für alle $m \vert l$, sodass $\text{ggT}(m,\frac{\ell}{m}) = 1$.
		\end{enumerate}
	\end{definition}
\end{framed}

\begin{framed}
	\begin{lemma}\label{lem:mod}
		Ist $L$ ein gerades Gitter der Dimension $n$.
		\begin{enumerate}[label=(\roman*)]
			\item Ist $L$ $\ell$-modular, dann ist $\Det(L) = \ell^{\frac{n}{2}}$. Insbesondere muss daher $n$ gerade sein.
			\item Ist $L$ $\ell$-modular und $\ell$ quadratfrei, dann hat $L$ die Stufe $\ell$.
			\item Ist $L$ stark $\ell$-modular, von Stufe $\ell$ und $\ell$ quadratfrei, dann ist $L$ auch $\ell$-modular.
		\end{enumerate}
	\end{lemma}
\end{framed}

\begin{beweis}
	\begin{enumerate}[label=(\roman*)]
		\item Nach Bem. \eqref{bem:dual} ist $\Det(L^\#) = \Det(L)^{-1}$. Somit
			\begin{equation*}
				\Det(L) = \Det(\sqrt{\ell} L^\#) = \ell^n \Det(L^\#) = \frac{\ell^n}{\Det(L)}.
			\end{equation*}
			Also folgt die Behauptung.
		\item Sei $a$ die Stufe von $L$, dann ist $\sqrt{a}L^\#$ gerade und hat insbesondere eine ganzzahlige Determinante. Nach (i) erhalten wir $\Det(\sqrt{a}L^\#) = \left( \frac{a^2}{\ell} \right) ^\frac{n}{2} \stackrel{!}{\in} \Z$. Da $\ell$ quadratfrei sieht man also $\ell \vert a$. Andersherum ist $L \cong \sqrt{\ell}L^\#$, also selbstverständlich auch $\sqrt{\ell} L^\#$ gerade, somit $a \vert \ell$.
		\item $L$ hat quadratfreie Stufe $\ell$, also ist $\ell L^\# \subseteq L$ nach Lemma \eqref{lem:level}. Wir erhalten
			\begin{equation*}
				L \cong \sqrt{l} L^{\#,\ell} = \sqrt{\ell} \left( \frac{1}{\ell} L \cap L^\# \right) = \sqrt{\ell} L^\#.
			\end{equation*}
	\end{enumerate}
\end{beweis}

Quebbemann zeigte in \cite{quebbemann}, dass die Theta-Reihen eines modularen Gitters Modulform einer bestimmten Gruppe ist. Außerdem hat die Algebra der Modulformen eine besonders einfache Gestalt, wenn die Summe der Teiler von $\ell$ selbst ein Teiler von $24$ ist. Konkret ist diese Eigenschaft für $\ell \in \lbrace 1,2,3,5,6,7,11,14,15,23 \rbrace$ erfüllt. In der Literatur sind diese Stufen also besonders interessant. Es lässt sich zeigen (vgl. z.B. \cite[1.2.2]{juergens}), dass der Raum der Modulformen der erwähnten Gruppe in diesem Fällen ein eindeutiges Element $\theta$ der Form $1 + O(q^d)$ mit möglichst großem $d$ und ganzzahligen Koeffizienten hat. Wir wollen den Begriff eines \textit{extremalen Gitters} definieren als ein Gitter, welches ein möglichst großes Minimum besitzt, also ein Gitter mit Thetareihe $\theta$. In unseren Spezialfällen gilt $d = 1 + \lfloor \frac{n}{k_1} \rfloor$, wobei $k_1$ Tabelle \eqref{tab:k1} zu entnehmen ist.

\begin{table}
	\centering
	\begin{tabular}{c|c|c|c|c|c|c|c|c|c|c}
		$\ell$	&$1$	&$2$	&$3$	&$5$	&$6$	&$7$	&$11$	&$14$	&$15$	&$23$\\
		\hline
		$k_1$		&$24$	&$16$	&$12$	&$8$	&$8$	&$6$	&$4$	&$4$	&$4$	&$2$
	\end{tabular}
	\caption{$k_1$ Werte nach $\ell$.\label{tab:k1}}
\end{table}

Wir können also definieren:

\begin{framed}
	\begin{definition}
		Sei $L$ ein $\ell$-modulares Gitter der Dimension $n$ und $\ell \in \lbrace 1,2,3,5,6,7,11,14,15,23 \rbrace$. Erfüllt $L$ die Schranke
		\begin{equation*}
			\Min(L) \geq 2 \left( 1 + \lfloor \frac{n}{k_1} \rfloor \right)
		\end{equation*}
		wobei $k_1$ gewählt ist wie in Tabelle \eqref{tab:k1}, so nennen wir $L$ ein \textit{extremales Gitter}.
	\end{definition}
\end{framed}

Die Dimensionen, welche jeweils echt von $k_1$ geteilt werden bezeichnet man häufig auch als \textit{Sprungdimensionen}, da in diesen Fällen das Minimum im Vergleich zur nächst kleineren Dimension um $2$ nach oben $"$springt$"$.\\
Da die Determinante für $\ell$-modulare Gitter in fester Dimension nach Lemma \eqref{lem:mod} eindeutig bestimmt ist, liefern modulare Gitter mit möglichst großem Minimum die dichtesten Kugelpackungen. In diesem Sinne ist die Klassifikation extremaler Gitter besonders interessant. Defininiert man für ein $n$-dimensionales Gitter $L$ den Wert
\begin{equation}
	\gamma(L) := \frac{\Min(L)}{\Det(L)^\frac{1}{n}}
\end{equation} (die \textit{Hermite-Konstante}), so bedeutet eine größerer Wert für $\gamma$ ein dichteres Gitter (vgl. \cite[(1.5)]{conway}). Vergleicht man die hypothetischen Hermite-Konstanten extremaler Gitter (deren Existenz bisher nach \cite{juergens} noch nicht ausgeschlossen wurde) mit denen der dichtesten bisher bekannten Gitter (zu finden in \cite{database}), so fällt auf, dass die Entdeckung extremaler Gitter in den folgenden Stufen und Dimensionen $\leq 48$ jeweils neue dichteste Kugelpackungen liefern würden:
\begin{itemize}
	\item $3$-modular und $36$-dimensional
	\item $3$-modular und $38$-dimensional
	\item $5$-modular und $32$-dimensional
	\item $5$-modular und $36$-dimensional
	\item $5$-modular und $40$-dimensional
	\item $5$-modular und $44$-dimensional
	\item $5$-modular und $48$-dimensional
	\item $6$-modular und $40$-dimensional
	\item $7$-modular und $32$-dimensional
	\item $7$-modular und $34$-dimensional
	\item $7$-modular und $38$-dimensional
	\item $7$-modular und $40$-dimensional
	\item $7$-modular und $46$-dimensional
	\item $11$-modular und $18$-dimensional
	\item $11$-modular und $22$-dimensional
	\item $14$-modular und $28$-dimensional
	\item $15$-modular und $28$-dimensional
\end{itemize}
Wie man sieht, ist die Erforschung extremaler modularer Gitter also von großem Interesse für die Gittertheorie. Im nächsten Kapitel beschreiben wir nun eine Vorgehensweise, modulare Gitter zu klassifizieren, welche zusätzlich eine Struktur als gebrochenes Ideal eines Zahlkörpers aufweisen, sogenannte \textit{Ideal-Gitter}.

%===================================================================
% Kapitel 3: Idealgitter
%===================================================================
\chapter{Ideal-Gitter}

\section{Definitionen}

Wir geben nun die Definition eines Ideal-Gitter abgeleitet aus \cite{fluckiger} an.

\begin{framed}
	\begin{definition}
		\begin{enumerate}[label=(\roman*)]
			\item Ein \textit{(algebraischer) Zahlkörper} ist eine endliche Erweiterung des Körpers $\Q$.
			\item Der \textit{Ring der ganzen Zahlen} eines Zahlkörpers $K$ ist der Ring
				\begin{equation*}
					\Z_K := \lbrace a \in K \mid \mu_{a, \Q}(X) \in \Z\left[X\right] \rbrace.
				\end{equation*}
			\item Die \textit{Norm} eines Ideals $\I$ von $\Z_K$ ist definiert als
				\begin{equation*}
					\No(\I) := \vert \Z_K / \I \vert.
				\end{equation*}
			\item Ein Zahlkörper $K$ heißt \textit{CM-Körper}, falls $K$ total-imaginär ist und ein total-reeller Teilkörper $K^+ \leq K$ existiert mit $\left[ K : K^+ \right] = 2$. 
			\item Sei $K$ ein CM-Körper und $\Z_K$ der Ring der ganzen Zahlen in $K$. Ein \textit{Ideal-Gitter} ist ein Gitter $(\I,b)$, sodass $\I$ ein gebrochenes $\Z_K$-Ideal ist und\linebreak
				$b:\I \times \I \rightarrow \R$ eine symmetrische positiv-definite Bilinearform mit $b(\lambda x, y) = b(x, \overline{\lambda} y)$ für $x,y \in \I$ und $\lambda \in \Z_K$. Die Abbildung $\bar{\ }$ bezeichnet dabei die herkömmliche komplexe Konjugation.
			\item Ein Element $\alpha \in K^+$ heißt \textit{total-positiv}, wenn $\iota(\alpha) > 0$ für alle Einbettungen $\iota : K^+ \hookrightarrow \R$. Wir schreiben dann auch $\alpha \gg 0$. Die Menge aller total-positiven Elemente in $K^+$ wird mit $K^+_{\gg 0}$ bezeichnet.
		\end{enumerate}
	\end{definition}
\end{framed}

Bis auf weiteres sei im Folgenden stets $K$ ein CM-Körper, $\Z_K$ der Ring der ganzen Zahlen in $K$ und $K^+$ der maximale total-reelle Teilkörper von $K$.

\begin{bemerkung}
	Die Eigenschaften der Bilinearform in der obrigen Definition sind nach \cite{fluckiger} äquivalent dazu, dass ein total-positives Element $\alpha \in K^+$ existiert mit $b(x,y) = \text{Spur}_{K/\Q}(\alpha x \overline{y})$. Wir können Ideal-Gitter daher auch durch die Notation $(\I, \alpha)$ beschreiben.\\
	Ein Ideal-Gitter $\I$ kann immer auch als $\Z$-Gitter betrachtet werden, indem man $\Z_K$-Erzeuger von $\I$ und eine $\Z$-Basis von $\Z_K$ zu $\Z$-Erzeugern von $\I$ kombiniert. Im Folgenden bezeichnen wir daher $\I$ als gerade, ganz, modular, etc., falls $\I$ als $\Z$-Gitter diese Eigenschaften erfüllt und $\I^\#$ als das Dualgitter von $\I$ als $\Z$-Gitter.
\end{bemerkung}

Wir beschäftigen uns in dieser Arbeit mit Ideal-Gittern über zyklotomischen Zahlkörpern, also Körpern der Form $\Q(\zeta_m)$ für primitive $m$-te Einheitswurzeln $\zeta_m$. Solche Körper sind CM-Körper mit maximalem total-reellem Teilkörper $K^+ = \Q(\zeta_m + \overline{\zeta_m})$. Wir erhalten Körper dieser Form, indem wir Automorphismen von $\Z$-Gittern betrachten, die wie primitive Einheitswurzeln operieren. Diese Aussagen wollen wir nun ein wenig präzisieren. Dazu eine kurze Definition:

\begin{framed}
	\begin{definition}
		Sei $K$ ein Körper und $m \in \N$. 
		\begin{enumerate}
			\item Ein Element $\zeta \in K$ heißt primitive $m$-te Einheitswurzel, falls $\vert \langle \zeta \rangle \vert = m$ ist.
			\item Gilt $\text{char}(K) \nmid m$ und sind $\zeta_1, \dots, \zeta_n$ die primitiven $m$-ten Einheitswurzeln in einem Zerfällungskörper von $X^m - 1$, dann heißt das Polynom
				\begin{equation*}
					\Phi_m(X) := \prod_{i=1}^n (X - \zeta_i)
				\end{equation*}
				das \textit{$m$-te Kreisteilungspolynom}.
		\end{enumerate}
	\end{definition}
\end{framed}

Einige wichtige bekannte Fakten zu Kreisteilungspolynomen (z.B. zu finden in \cite[Kap. 1]{Mollin}), sind die folgenden:

\begin{framed}
	\begin{satz}
		\begin{enumerate}[label=(\roman*)]
			\item Gilt $\text{char}(K) \nmid m$, so enthält der Zerfällungskörper von $X^m-1$ genau $\varphi(m)$ primitive $m$-te Einheitswurzeln. Dabei ist $\varphi(m) = \vert \left( \Z / m \Z \right)^\ast \vert$ die \textit{Eulersche $\varphi$-Funktion}.
			\item Ist $\text{char}(K) = 0$, dann ist $\Phi_m \in \Z \left[ X \right]$ und $X^m-1 = \prod_{d \vert m} \Phi_d$.
			\item Speziell für $K = \Q$ gilt $\left[ \Q \left( \zeta_m \right) : \Q \right] = \varphi (m)$ und $\Phi (m) \in \Q \left[ X \right]$ ist irreduzibel.
			\item Gilt $\text{char}(K) \nmid m$, so ist $K(\zeta_m) / K$ ist eine Galoiserweiterung.
		\end{enumerate}
	\end{satz}
\end{framed}

Wir sehen also, dass $\zeta_m$ genau dann eine primitive $m$-te Einheitswurzel ist, wenn sie das Minimalpolynom $\Phi_m$ hat. Wir können $\Z$-Gitter somit auf die folgende Weise als Ideal-Gitter auffassen (vgl. \cite[Abschnitt (5.2)]{nebe}):

\begin{framed}
	\begin{lemma}
		Sei $L$ ein $\Z$-Gitter in einem bilinearen Vektorraum $(V,b)$ und $\sigma \in \text{Aut}(L)$ mit $\mu_\sigma = \Phi_m$ für ein $m \in \N$. Dann ist $L$ isomorph zu einem Ideal-Gitter in $\Q \left( \zeta_m \right)$.
	\end{lemma}
\end{framed}

\begin{beweis}
	Durch die Operation von $\sigma$ wird $\Q L$ mittels $\zeta_m \cdot x := \sigma(x)$ für $x \in \Q L$ zu einem eindimensionalen $\Q \left( \zeta_m \right)$-Vektorraum und $L$ zu einem ein $\Z \left[ \zeta_m \right]$-Modul. Wegen $\Z \left[ \zeta_m \right] = \Z_{\Q\left[\zeta_m \right]}$ ist $L$ also ein gebrochenes Ideal in $\Q \left( \zeta_m \right)$.\\
	Da $\sigma$ ein Automorphismus ist, ist die Bilinearform $b:L \times L \rightarrow \Q$ des Vektorraums $\zeta_m$-invariant. Sei nun $\lambda \in \Z \left[ \zeta_m \right]$ beliebig. Wir können $\lambda = \sum_{i=0}^{m-1} a_i \zeta_m^i $ für Koeffizienten $a_i \in \Z$ schreiben und sehen
	\begin{equation*}
		b(\lambda x, y) = \sum_{i=0}^{m-1} a_i b(\zeta_m^i x, y) = \sum_{i=0}^{m-1} a_i b(x, \zeta_m^{-i} y) = \sum_{i=0}^{m-1} a_i b(x, \overline{\zeta_m^i} y) = b(x, \overline{\lambda} y),
	\end{equation*}
	womit die Eigenschaften eines Ideal-Gitters erfüllt sind.
\end{beweis}

Mittels der Klassifikation der Ideal-Gitter über $\Q(\zeta_m)$ erhalten wir also zugleich alle  $\Z$-Gitter mit Minimalpolynom $\Phi_m$. Wie diese Klassifikation durchgeführt werden kann, erläutern wir in den nächsten Abschnitten.

\section{Strategie zur Klassifikation}

Die in den nächsten Abschnitten beschriebenen Aussagen und Vorgehensweisen zur Klassifikation von Ideal-Gittern sind an \cite[Abschnitt (3.2)]{juergens} und \cite[Abschnitt (5.2)]{nebe} angelehnt.

\begin{framed}
	\begin{definition}
		Das $\Z_K$-ideal
		\begin{equation*}
			\Delta := \lbrace x \in K \mid \text{Spur}_{K/\Q}(x \overline{y}) \in \Z \text{ für alle } y \in \Z_K \rbrace
		\end{equation*}
		bezeichnet die \textit{inverse Differente} von $\Z_K$.
	\end{definition}
\end{framed}

Wir können nun das Dual eines Idealgitters mithilfe der inversen Differente ausdrücken.

\begin{framed}
	\begin{lemma}\label{lem:idealdual}
		Sei $(\I, \alpha)$ ein Ideal-Gitter. Dann ist $\I^\# = \overline{\I}^{-1} \Delta \alpha^{-1}$ das Dualgitter von $\I$ als $\Z$-Gitter.
	\end{lemma}
\end{framed}

\begin{beweis}
	\begin{align*}
	\I^\# &= \lbrace x \in K \mid b(x,\I) \subseteq \Z \rbrace\\
	&= \lbrace x \in K \mid \text{Spur}_{K/\Q}(\alpha x \overline{\I}) \subseteq \Z \rbrace\\
	&= \alpha^{-1} \lbrace x \in K \mid \text{Spur}_{K/\Q}(x \overline{\I}) \subseteq \Z\rbrace\\
	&= \alpha^{-1} \overline{\I}^{-1} \lbrace x \in K \mid \text{Spur}_{K/\Q}(x \overline{\Z_K}) \subseteq \Z\rbrace\\
	&= \overline{\I}^{-1} \Delta \alpha^{-1}.
	\end{align*}
\end{beweis}

Mit Blick auf modulare Gitter kann man damit die nächste Folgerung ziehen:

\begin{framed}
	\begin{korollar}
		Sei $\ell$ quadratfrei und $(\I, \alpha)$ ein ganzes Ideal-Gitter der Stufe $\ell$. Die Menge $\B := \alpha \I \overline{\I} \Delta^{-1}$ ist ein $\Z_K$-Ideal mit $\ell \Z_K \subseteq \B$ und Norm $\No(\B) = \text{det}(\I)$.
	\end{korollar}
\end{framed}

\begin{beweis}
	Da $\ell$ quadratfrei ist, gilt $\ell \I^\# \subseteq \I$ nach Lemma \eqref{lem:level}. Mit Lemma \eqref{lem:idealdual} bedeutet dies:
	\begin{align*}
		&\ell \I^\# \subseteq \I \subseteq \I^\#\\
		\Leftrightarrow\quad& \ell \overline{\I}^{-1} \Delta \alpha^{-1} \subseteq \I \subseteq \overline{\I}^{-1} \Delta \alpha^{-1}\\
		\Leftrightarrow\quad& \ell \Z_K \subseteq \alpha \I \overline{\I} \Delta^{-1} \subseteq \Z_K.
	\end{align*}
	Für die Norm gilt
	\begin{equation*}
		\text{det}(\I) = \vert \I^\# / \I \vert = \vert \Z_K / \left(\I\left(\I^\#\right)^{-1}\right) \vert = \vert \Z_K / \B \vert = \No(\B).
	\end{equation*}
\end{beweis}

Da es jeweils nur endlich viele $\Z_K$-Ideale mit bestimmter Norm gibt, existieren bei der Konstruktion von Idealgittern mit fester Determinante nur endlich viele Möglichkeiten für $\B$. Konkret wird unsere Strategie im groben daraus bestehen, alle (relevanten) Möglichkeiten für $\I$ und $\B$ durchzugehen und zu testen, für welche davon das Ideal $\left( \I \overline{\I} \right)^{-1} \Delta \B$ ein Hauptideal mit total-positivem Erzeuger $\alpha \in K^+$ ist. Dazu machen wir zunächst einige Einschränkungen, um den Suchraum zu verkleinern.

\section{Die Klassengruppe}

\begin{framed}
	\begin{definition}
		Die \textit{Klassengruppe} 
		\begin{equation*}
			\text{Cl}_K := \lbrace J \mid J \text{ ist gebrochenes } \Z_K \text{-Ideal} \rbrace / \lbrace (c)_{\Z_K} \mid c \in K^\ast \rbrace.
		\end{equation*}
	\end{definition}
\end{framed}


\begin{framed}
	\begin{lemma}\label{lem:squares}
		Seien $\I$ ein gebrochenes $\Z_K$-Ideal und $\alpha \in K^+_{\gg 0}$. Für $\lambda \in K^\ast$ gilt $(\lambda \I, \alpha) \cong (\I, \lambda \overline{\lambda}\alpha)$.
	\end{lemma}
\end{framed}

\begin{beweis}
	Sei $b_\alpha : K \times K \rightarrow \R, (x,y) \mapsto \text{Spur}_{K/\Q} (\alpha x \overline{y})$ die zu $\alpha$ gehörige Bilinearform. Dann ist
	\begin{equation*}
		b_\alpha(\lambda x, \lambda y) = \text{Spur}_{K/\Q} (\lambda \overline{\lambda} \alpha x \overline{y}) = b_{\lambda \overline{\lambda} \alpha} (x,y).
	\end{equation*}
	Folglich ist $\psi : (K, b_{\lambda \overline{\lambda }\alpha}) \rightarrow (K, b_\alpha), x \mapsto \lambda x$ eine Isometrie mit $\psi(\I) = (\lambda \I)$.
\end{beweis}

Mit dieser Aussage genügt es also, aus jeder Klasse der 
jeweils nur einen Vertreter zu betrachten. Wählt man $\lambda \in \Z_K^\ast$, so zeigt das Lemma, dass $(\I, \alpha) \cong (\I, \lambda \overline{\lambda} \alpha$. Für $\alpha$ reichen also Vertreter modulo $\lbrace \lambda \overline{\lambda} \mid \lambda \in \Z_K^\ast \rbrace$.\\
Wir wollen nun die zu untersuchenden Möglichkeiten für $\I$ noch weiter einschränken: Ist $K / \Q$ galoissch (wie es für zyklotomische Zahlkörper der Fall ist), so genügt ein Repräsentant modulo der Operation der Galosgruppe.

\begin{framed}
	\begin{lemma}
		Sei $K / \Q$ eine Galoiserweiterung, $\I$ ein gebrochenes $\Z_K$-Ideal und $\alpha \in K^+_{\gg 0}$. Für $\sigma \in \text{Gal}(K/\Q)$ ist $(\I, \alpha) \cong \left( \sigma(\I), \sigma(\alpha) \right)$.
	\end{lemma}
\end{framed}

\begin{beweis}
	Da die Spur invariant unter der Galoisgruppe ist, erhält man die folgende Gleichungskette.
	\begin{align*}
		b_{\sigma(\alpha)} (\sigma(x), \sigma(y)) &= \text{Spur}_{K/\Q} \left( \sigma(\alpha) \sigma(x) \overline{\sigma(y)}\right)\\
		&= \text{Spur}_{K/\Q} \left( \sigma\left( \alpha x \overline{y} \right) \right) = \text{Spur}_{K/\Q} \left( \alpha x \overline{y} \right) = b_\alpha (x,y)
	\end{align*}
	Also induziert $\sigma$ eine Isometrie $\sigma : (K, b_\alpha) \rightarrow (K, b_{\sigma(\alpha)})$.
\end{beweis}

\section{Total-positive Erzeuger}

Wir benötigen nun einen Test, welcher für ein gegebenes gebrochenes $\Z_K$-Ideal $\I$ überprüft, dieses von einem total-positiven Element $\alpha \in K^+_{\gg 0}$ erzeugt wird. Dazu untersuchen wir zuerst, ob $\I$ überhaupt von einem Element aus $K^+$ erzeugt ist und anschließend, wann ein Ideal $\alpha'\Z_K$ für $\alpha' \in K^+$ einen total-positiven Erzeuger hat.

\begin{framed}
	\begin{satz}
		Sei $\I$ ein gebrochenes $\Z_K$-Ideal. Es existiert genau dann ein $\alpha' \in K^+$ mit $\I = \alpha' \Z_K$, wenn $\I \cap K^+ = \alpha' \Z_{K^+}$ und für jeden Primteiler $\p$ von $\I$ gilt
		\begin{itemize}
			\item Ist $\p$ verzweigt in $K / K^+$, so ist $\nu_\p(\I) \in 2\Z$.
			\item Ist $\p$ unverzweigt in $K / K^+$, so ist $\nu_\p(\I) = \nu_{\overline{\p}}(\I)$.
		\end{itemize}
	\end{satz}
\end{framed} 

\begin{beweis}
	Wir zeigen zunächst, dass die Bedingungen an die Primteiler äquivalent dazu sind, dass $\I = (\I \cap K^+)\Z_K$.\\
	Sei dazu zuerst $\I = (\I \cap K^+)\Z_K$ erfüllt. Seien
	\begin{equation*}
		\I \cap K^+ = \prod_\a \a^{\nu_\a(\I \cap K^+)},\quad \I = \prod_\p \p^{\nu\p(\I)}
	\end{equation*} 
	die Primidealzerlegungen. Dann folgt
	\begin{equation*}
		\prod_\a \a^{\nu_\a(\I \cap K^+)} \Z_K = \prod_\p \p^{\nu\p(\I)}.
	\end{equation*}
	Aufgrund der Eindeutigkeit der Primidealzerlegung bedeutet dies
	\begin{equation*}
		\a^{\nu_\a(\I \cap K^+)} \Z_K = \prod_{\p\vert \a} \p^{\nu\p(\I)}
	\end{equation*}
	für jedes Primideal $\a$ von $\Z_{K^+}$. Es ist $\left[ K : K^+ \right] = 2$, also kann $\a\Z_K$ nur eine der Formen $\p$, $\p^2$, oder $\p\overline{\p}$ für ein Primideal $\p$ in $\Z_K$ annehmen.
	\begin{itemize}
		\item Falls $\a\Z_K = \p^2$ (also falls $\p$ verzweigt ist), so folgt $\nu_\p(\I) = 2 \nu_a(\I \cap K^+) \in 2 \Z$.
		\item In den anderen beiden Fällen (also falls $\p$ unverzweigt ist) gilt\linebreak
		$\nu_\p(\I) = \nu_\a(\I \cap K^+) = \nu_\p(\overline{\I})$.
	\end{itemize}
	Seien nun andersherum die Primideal-Bedingungen erfüllt. Definiert man
	\begin{equation*}
		\a := \prod_\b \b^{\nu_\b(\a)}, \quad \nu_\b(\a) = \begin{cases} \nu_\p(\I) &\b \Z_K \in \lbrace \p, \p \overline{\p} \rbrace\\ \frac{1}{2}\nu_\p(\I) &\b\Z_K = \p^2 \end{cases}
	\end{equation*}
	so gilt
	\begin{equation*}
		\I = \prod_\p \p^{\nu_\p(\I)} = \prod_\b \left( \b \Z_K \right)^{\nu_\b(\a)} = \a \Z_K.
	\end{equation*}
	Also folgt $(\I \cap K^+)\Z_K = (\a \Z_K \cap K^+)\Z_K = \a \Z_K = \I$ und es gilt die behauptete Äquivalenz.\par
	Die Behauptung des Satzes wurde somit darauf reduziert, dass genau dann $\I = \alpha' \Z_K$, wenn $\I \cap K^+ = \alpha' \Z_{K^+}$ und $\I = (\I \cap K^+)\Z_K$ für $\alpha' \in K^+$. Dies folgt allerdings leicht mithilfe von $(\alpha' \Z_K) \cap K^+ = \alpha'\Z_{K^+}$.
\end{beweis}

\begin{framed}
	\begin{lemma}\label{lem:sig}
		Sei $\alpha' \in K^+$. Ein total-positives Element $\alpha \in K_{\gg 0}^+$ ist genau dann ein Erzeuger des Ideals $\alpha' \Z_K$, wenn eine Einheit $\epsilon \in \Z_{K^+}^\ast$ existiert mit $\alpha = \alpha' \epsilon$ und\linebreak
		$\text{sign}\left((\iota(\epsilon)\right) = \text{sign}\left( \iota(\alpha')\right)$ für alle Einbettungen $\iota : K^+ \hookrightarrow \R$.
	\end{lemma}
\end{framed}

\begin{beweis}
	Ein weiterer Erzeuger hat immer die Gestalt $\alpha = \alpha' \epsilon$ für eine Einheit $\epsilon \in (\Z_{K^+})^\ast$. Damit $\alpha$ total-positiv wird muss für alle Einbettungen $\iota : K^+ \hookrightarrow \R$ gelten:
	\begin{equation*}
		1 \stackrel{!}{=} \text{sign}\left( \iota(\alpha) \right) = \text{sign}\left( \iota(\alpha') \iota(\epsilon) \right) = \text{sign}\left( \iota(\alpha') \right) \text{sign}\left( \iota(\epsilon) \right)
	\end{equation*}
	Also müssen die Vorzeichen jeweils identisch sein.
\end{beweis}

Elemente aus $\left(\Z_{K^+}^\ast\right)^2$ haben immerzu positives Signum bezüglich allen Einbettungen. Außerdem liefern total-positive Elemente, die in der gleichen Klasse modulo Quadraten liegen nach Lemma \eqref{lem:squares} isomorphe Idealgitter. Es genügt also, sich bei der Suche nach einer Einheit wie im vorherigen Lemma auf Vertreter modulo Quadraten zu beschränken. Nach dem Dirichletschen Einheitensatz \cite[Theorem (7.4)]{neukirch} hat die Einheitengruppe die Struktur
\begin{equation*}
	\Z_{K^+}^\ast = \lbrace \pm 1 \rbrace \times \Z^{t-1}.
\end{equation*}

mit $t := \left[K^+ : \Q \right]$. Die Erzeuger $(\epsilon_1, \dots \epsilon_t)$ der Gruppe heißen  \textit{Grundeinheiten}. Jede Einheit $\epsilon$ lässt sich also darstellen in der Form $\epsilon = \epsilon_1^{\nu_1} \dots \epsilon_t^{\nu_t}$. Das folgende Korollar liefert uns nun die Lösung auf unsere Frage nach den total-positiven Erzeugern.


Dann lässt sich folgendes Korollar ziehen:

\begin{framed}
	\begin{korollar}
		Sei $\alpha' \in K^+$, seien die Einbettungen von $K^+$ in $\R$ gegeben durch $\iota_1, \dots, \iota_{t}$ und seien $\epsilon_1, \dots, \epsilon_t$ die Grundeinheiten von $\Z_{K^+}^\ast$. 				Definiere die Matrix
		\begin{equation*}
			M \in \F_2^{t \times t}, \quad M_{ij} = \begin{cases} 1 &, \text{sign}(\iota_i(\epsilon_j) = -1\\ 0 &, \text{sign}(\iota_i(\epsilon_j) = 1\end{cases}
		\end{equation*}
		und den Vektor
		\begin{equation*}
			V \in \F_2^{1 \times t}, \quad V_i = \begin{cases} 1 &, \text{sign}(\iota_i(\alpha') = -1\\ 0 &, \text{sign}(\iota_i(\alpha') = 1\end{cases}.
		\end{equation*}
		Dann sind die total-positiven Erzeuger des Ideals $\alpha' \Z_K$ genau die Elemente der Menge $\lbrace \alpha' \epsilon_1^{x_1} \dots \epsilon_t^{x_t} \epsilon^2 \mid x \in \F_2^{1 \times t}, x M = V, \epsilon \in (\Z_{K^+}^\ast) \rbrace$.
	\end{korollar}
\end{framed}

\begin{beweis}
	Nach Lemma \eqref{lem:sig} und da Quadrate immerzu positives Signum haben, sind die total-positiven Erzeuger gegeben durch die Elemente $u = \alpha' \epsilon_1^{x_1} \dots \epsilon_t^{x_t} \epsilon^2$, wobei $x \in \F_2^{1 \times t}$, $\epsilon \in \Z_{K^+}^\ast$ und $\epsilon_1^{x_1} \dots \epsilon_t^{x_t}$ bezüglich allen Einbettungen dasselbe Signum wie $\alpha'$ hat. Das Signum bezüglich einem $\iota_i$ ist genau dann gleich, wenn
	\begin{equation*}
		\vert \lbrace j \mid \text{sign}(\iota_i(\epsilon_j)) = -1 \text{ und } x_j = 1 \rbrace \vert \equiv \begin{cases} 1 \ (\text{mod } 2) &,\text{sign}(\iota_i(\alpha')) = -1 \\ 0 \ (\text{mod } 2) &,\text{sign}(\iota_i(\alpha')) = 1 \end{cases}.
	\end{equation*}
	Diese Kongruenz ist aber genau dann erfüllt, wenn $x$ Lösung des linearen Gleichungssystems $xM = V$ ist.
\end{beweis}

\begin{bemerkung}
	Um später in der Implementierung Zeit zu sparen, kann man bemerken, dass sich verschiedene total-positive Erzeuger des gleichen Ideals jeweils lediglich um eine total-positive Einheit unterscheiden. Es lohnt sich also, zu Beginn des Algorithmus die Menge aller total-positiven Einheiten (also die Elemente im Kern von $M$) zu berechnen, sodass man später pro Ideal jeweils nur eine spezielle Lösung des Gleichungssystems finden muss und die Menge aller total-positiven Erzeuger durch Multiplikation mit den vorher berechneten total-positiven Einheiten erstellt.
\end{bemerkung}

%===================================================================
% Literatur
%===================================================================
\newpage
\bibliography{Quellen}

\end{document}
